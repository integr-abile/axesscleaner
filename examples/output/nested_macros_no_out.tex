\documentclass[11pt,reqno]{amsart}
\providecommand{\opt}[1]{\framebox[3ex][c]{#1}} 
\DeclareMathOperator{\dom}{dom}  
\DeclareMathOperator{\re}{Re}  
\DeclareMathOperator{\im}{Im}  
\newcommand{\erre}{\mathbb{R}} 
\newcommand{\ci}{\mathbb{C}} 
\DeclareMathOperator{\sign}{sign}  
\DeclareMathOperator{\de}{d\!} 
\DeclareMathOperator{\res}{Res} 
\newcommand{\F}{\mathcal{F}} 
\renewcommand{\L}{\mathcal{L}} 
\newcommand{\LL}{\L^2} 
\newcommand{\LLL}{(\L^2+(\LL)^3} 
\newcommand{\zaa}{\alpha}
\newcommand{\zg}{\gamma}
\newcommand{\weird}[3]{\sum\sb {n = #1}^{#2} \F(#3) - 7 +\frac{#1}{#2}}
\def\indicator{\mathbf{1}}
\begin{document}
\noindent
Student's surname and name \underline{\hspace{68.5ex}}
\vspace{1.5ex}
\noindent
Student's number \underline{\hspace{80ex}}
\vspace{8ex}
This formula is wird:
\[\mathcal{L}(3) + \mathcal{F}(4) + \mathcal{L}^2(6) + ((\mathcal{L}^2+(\mathcal{L}^2)^3)^7 \mathbf{1}\sb {5}\]
This one is actually worst:
\[\lim\sb {x\to\alpha} \gamma=\log\sb a\sb r \sum\sb {n\ =\ \frac{1}{\{\mathcal{L}^2\}}}^{a}\ \mathcal{F}(\alpha)\ -\ 7\ +\frac{\frac{1}{\{\mathcal{L}^2\}}}{a}\ d\]
\noindent
\end{document}
